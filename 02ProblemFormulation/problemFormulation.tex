\section{Problem formulation} \label{problemFormulation}
The motivation for this project comes from the frustration many employees have, when accessing the premises of an enterprise because of an access card they have to carry around, as well as the need to remember passwords to access corporate account and associated services.
On the other side, there are enterprises which need to keep up with the latest technologies in order to protect their resources, stopping the use of passwords and using more intuitive access system is one of methods. Furthermore, having only one system endpoint from which all access levels and management can be done is very convenient for them.

Taking into account the challenges and issues mentioned previously, guided us in creation of the scope of the project and definition of the actual problem to be solved. The problem formulation is as follows:

\begin{center}
    \textit{“How to design Access control system, combining both, the physical on-premise and online access to resources managed in one system?”}
\end{center}

\paragraph{Sub-questions}
Keeping on track during the design, development and implementation is often hard but important thing to do. Therefore, defining sub-questions and taking small pieces of the problem one by one, instead of the whole problem is a convenient approach. Problem areas recognized in the project are:

ROUGH DRAFT - QUESTIONS
\begin{itemize}[noitemsep]
    \item What solutions are there on the market?
    \item What is the system’s architecture?
    \item Which technologies are the best for proposed system?
    \item Which components of the system must be implemented for MVP?
    \item What are the requirements of the system?
\end{itemize}