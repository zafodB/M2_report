\subsection{Prototype Requirements} \label{sec:design-prototype-requirements}

Based on the use cases highlighted in the previous section, we describe the requirements necessary to implement these use cases. Naturally, all requirements linked with given use cases should be implemented. The following requirements are linked to the selected use cases: RQ-1, RQ-2, RQ3, RQ-4, RQ-5, RQ-18 and RQ-19. These requirements are implemented in the prototype and drive the design choices behind the system architecture. 

% TODO describe selected Requirements

% \begin{table}[H]
%     \footnotesize
%     \onehalfspacing
%     \centering
%     \begin{tabular}{|c|p{15cm}|}
%     \hline
%     \cellcolor[HTML]{CBCEFB}\textbf{ID}&\cellcolor[HTML]{CBCEFB}\textbf{Description}\\
%     \hline
%     RQ-1&User must be able to authenticate with single-factor cryptographic device based on FIDO2 to use their enterprise digital identity.\\
%     \hline
%     RQ-2&User must be able to authenticate with smartphone using FIDO2 flow to use their enterprise digital identity.\\
%     \hline
%     RQ-4&User must be able to sign-in to external client system with their enterprise digital identity.\\
%     \hline
%     RQ-5&User must be able to open a door with their enterprise digital identity.\\
%     \hline
    %     % RQ-19&\textbf{External client system must be able to access protected resources, when authorised to do so.}\\
%     RQ-18&External client system must be able to access protected resources if permission granted.\\
%     \hline
%     RQ-19&External client system must be able to obtain grant permission.\\
%     \hline
%     \hline
%     RQ-20&Authentication to PACS using enterprise digital identity must be comparably fast to traditional RFID access cards (order of milliseconds to seconds).\\
%     \hline
%     % RQ-21&\textbf{External system must be able to access protected resources.}\\
%     % \hline
%     RQ-21&The system must use OIDC + OAuth for Authentication and Authorisation.\\
%     \hline
%     RQ-22&The system must use XACML for Identity and Access management\\
%     \hline
%     \end{tabular}
%     \caption{List of functional requirements \#1, \#2, \#4, \#5, \#18, \#19 and non-functional requirements \#20, \#21, \#22 for the prototype.}
%     \label{tab:prototype-requirements}
% \end{table}