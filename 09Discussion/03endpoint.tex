\subsubsection*{Coupling}
Section~\ref{sec:design-components} defines a number of individual components in the system, each component responsible for it's own, narrow task. A \acrshort{pdp} (only) evaluates policies, a User Directory (only) answers queries about users and the UserInfo endpoint only serves information about users to the external systems. This design represents a \textit{loosely coupled} system, where each module focuses only on its own task and is independent from its neighbours.

An alternative designed to the proposed system could be a single heavy-weight endpoint -- such as \acrshort{aaserver} -- that would take tasks of authenticating users, serving requests for user information and maybe even serving requests for protected resources. The advantages for external (and internal) applications would be that only a single endpoint is called at all times. Furthermore, only this single endpoint would be exposed to requests from the public internet, thus reducing the risks of security breach.

On the other hand, this approach suffers from the known disadvantages of tight coupling. The the single endpoint would be subject to bigger load and would need to be more versatile, than many separated endpoints. Furthermore, shall an outage occur to this single endpoint, the entire system would be inoperable, similarly, an update to a single functionality would require the update of the whole system. Loosely coupled, individual endpoints do not suffer from these pains. We maintain that benefits of the loose coupling outweigh its shortcomings and it presents a better fit in the access control scenario.