\subsubsection*{Multi-factor authentication considerations} \label{multiFA}
Choosing a right combination of authenticators is crucial for every system as it can either strengthen the security or if badly combined even weaken it. As described in Section~\ref{sec:basic-terms} there are three groups of authenticators.

\paragraph{Something you know}
Authenticators from this group has been used for centuries, however in the era of technical advance and modern technologies they become obsolete and hard to manage. People are using dozens of services online for which they need to remember some kind of password or \acrshort{pin}. Therefore, they are often choosing easy to remember passwords or are using the same for all services~\cite{2018OnlineSurvey}. On top of that, passwords are prone to phishing attacks and it is possible a user will forget it. 

\paragraph{Something you have}
In the enterprise environment, access cards (RFID or proximity cards) are used often, this is just one factor authentication thought. It is popular to use a phone to receive a SMS or an email with a \acrfull{otp} to prove identity, but it has been advised by \acrshort{nist} not to use it anymore~\cite{NIST2017NISTBlog}. On the other hand, using an \acrshort{otp} generator such as a smartphone application or a dedicated hardware device is feasible, because of its simplicity, ease of use and price.

\paragraph{Something you are}
People tend to forget and lose things; therefore, one can argue that using \textit{``something you are’’} is the most user friendly and secure of all above mentioned authenticators. Fingerprint scan is among the most used authenticators. FIDO2 which is implemented in our system has been certified on Android smartphones and biometric scanners can be used to authenticate users. Another option is to integrate fingerprint scanner to a physical cryptographic key. This way, the ownership of the device and identity of the user can be verified at the same time. Also, it fulfils the \acrshort{nist} requirement about using Biometrics in \acrshort{mfa}~\cite{Grassi2017DigitalManagement}. 

\paragraph{Suggested solution}
Each of the authenticator types has it pros and cons. The most important criteria for choosing another factor in \acrshort{mfa} are the security and ease of use. Based on the analysis in this project, we envision three candidate solutions that could be used to increase the security of the proposed \acrshort{acs} solution:
\begin{itemize}[noitemsep]
    \item \acrshort{otp} generator in form of calculator or smartphone application
    \item \acrshort{fido}2 physical cryptographic key with fingerprint scanner
    \item \acrshort{fido}2 certified smartphone with fingerprint scanner
\end{itemize}
All three solutions are secure~\cite{Grassi2017DigitalManagement, FIDOFIDO2Project}, fairly easy to use and can be implemented to our system.