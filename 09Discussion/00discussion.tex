\section{Discussion}\label{sec:discussion}
In this chapter we discuss the progress and the outcomes of the prototype implementation. We then consider several open problems that were not covered in this project, but are relevant for launch of a combined access control system on the market. Possible ways of performing multi-factor authentication are examined, followed by considerations about the future development of FIDO and its implementations. Lastly, potential future works that could extend and built on top of the system are discussed.

\subsection*{Implementation}
In Chapter~\ref{sec:design} we identify a number of use cases and requirements that should be implemented in the prototype to demonstrate the functionality of the system. During the implementation work it was confirmed that there is ongoing work on the different parts of the access control problem in the general coding community. This was demonstrated by the existence of numerous active repositories, libraries and guides for FIDO, OAuth and OIDC.

Despite existence of these, we were not able to implement all of the use cases set out for the prototype. A possible reason for this is the recentness of the involved technologies, mainly FIDO/Webauthn. While the specifications exists and are adopted by major browsers, there is still some degree of ambiguity among existing implementations in the ``gray'' areas not covered by the specification, such as how to store the data in a database, how to parse data to prepare them for transport, or what encoding version (\texttt{Base64} vs \texttt{Base64Url}) should be predominantly used.

In the OAuth/OIDC space, a lot of focus is put into integrating new applications with existing OAuth providers (implementing sign-in with Facebook, Google and similar). In commercial applications, it is not desired to implement own sign-in system and convergence towards the major providers can be seen. However, as demonstrated in the prototype, it is still possible to deploy a customised OAuth/OIDC Authorisation Server.

Even though the prototype does not implement all of the selected use cases, we find it valuable as it does demonstrate some of the fundamental flows of authentication and access policy enforcement. Further work on implementation would gradually resolve the existing issues and present a working solution. To arrive at a marketable \acrshort{mvp}, integration with Active Directory and/or \acrshort{ldap} would need to be implemented in addition to the other \acrshort{mvp} requirements specified in Section~\ref{sec:analysis-requirements}.

\subsection{Multi-factor authentication considerations} \label{multiFA}

Multi-factor authentication is an approach where two or more factors for successful authentication of a user are required. This way, if one of factors is compromised, there is a need for compromising the other factors as well in order to successfully access the resources. \acrshort{nist} in their Digital Identity Guidelines~\cite{Grassi2017Digital3} specifies three groups of authenticators which can be used for authentication:
\begin{itemize}[noitemsep]
    \item Something you know (typically a password, \acrshort{pin}, etc.)
    \item Something you have (typically a physical cryptographic key, authenticator application, etc.)
    \item Something you are (typically a fingerprint, retina scan, etc.)
\end{itemize}

Choosing a right combination of authenticators is crucial for every system as it can either strengthen the security or if badly combined even weaken it. Therefore, now we will analyse each of these three groups of authenticators and suggest the most suitable one.

\paragraph{Something you know}
Authenticators from this group has been used for centuries, however in the era of technical advance and modern technologies they become obsolete and hard to manage. People are using dozens of services online for which they need to remember some kind of password or \acrshort{pin}. Therefore, they are often choosing easy to remember passwords or are using the same for all services~\cite{2018OnlineSurvey}. On top of that, passwords are prone to phishing attacks and it is possible a user will forget it. Furthermore, one of the aims of the system presented in this project, is to avoid the use of passwords. Using\textit{ ``something you know’’} as another factor would therefore go against our initial goal of not using this element.

\paragraph{Something you have}
In the enterprise environment, access cards (RFID or proximity cards) are used often, this is just one factor authentication thought. These days, there are many options for authentication using \textit{``something you have’’}, some being more user friendly than the others. It is popular to use a phone to receive a SMS or an email with a \acrfull{otp} to prove identity, but it has been advised by NIST not to use it anymore~\cite{NIST2017NISTBlog}. For the purposes of our system, as mentioned before (in section \ref{sec:analysis-authentication}), we are planning to use a physical cryptographic key to authenticate a user as a primary authentication method, instead of \textit{``something you know’’}. Using two physical cryptographic keys is not advised because of a high chance of losing them both at the same time. On the other hand, using \acrshort{otp} generator such as smartphone application or smart card is feasible, because of its simplicity, ease of use and price. 

\paragraph{Something you are}
People tend to forget and lose things; therefore, one can argue that using \textit{``something you are’’} is the most user friendly and secure of all above mentioned authenticators. Fingerprint scan is among the most used authenticators. FIDO2 which is implemented in our system has been certified on Android smartphones and biometric scanners can be used to authenticate users. Another option is to integrate fingerprint scanner to a physical cryptographic key. This way, the ownership of the device and identity of the user can be verified at the same time. Also, it fulfils the NIST requirement about using Biometrics in \acrshort{mfa}: \textit{Biometrics SHALL be used only as part of multi-factor authentication with a physical authenticator (something you have).}~\cite{Grassi2017DigitalManagement}. Both of these options are suitable for our enterprise scenario and can be proposed to be used.

\paragraph{Suggested solution}
Each of the authenticator types has it pros and cons. The most important criteria for choosing another factor in \acrshort{mfa} are the security and ease of use. Based on analysis above, three candidate solutions are found:
\begin{itemize}[noitemsep]
    \item \acrshort{otp} generator in form of calculator or smartphone application
    \item \acrshort{fido}2 physical cryptographic key with fingerprint scanner
    \item \acrshort{fido}2 certified smartphone with fingerprint scanner
\end{itemize}
All three solution are secure~\cite{Grassi2017DigitalManagement}, \cite{FIDOFIDO2Project}, fairly easy to use and can be implemented to our system. Our preference are both options containing FIDO2 as it is being implemented in the system already. However, as \acrshort{mfa} is not part of the requirements neither for \acrshort{mvp} nor for prototype, we will not go further in analyzing which of candidate solutions is the most feasible to implement.
\subsubsection*{FIDO considerations}
Shortly before this project was submitted, support of FIDO was announced by Windows as another major operating system, following Android~\cite{Mehta2019WindowsPasswordless}. This is a clear sign of growing support of password-less login scenarios. While both mobile and desktop devices included support for biometric sign-in to the OS for a number of year, this technology was not typically used outside the OS login and to log in to a limited number of native applications that implemented the OS specific libraries for integration with the biometric devices on the device.

FIDO2 helps by removing the need for direct OS integration by exposing the CTAP2 protocol to any interested party via Webauthn. While FIDO2 in itself is not directed at biometric authentication, it brings forward an important abstraction layer to enable secure, yet convenient password-less log-in experience, whether the authenticator is a physical cryptographic key, or the device itself with integrated biometric capabilities. This abstraction can be exploited in the proposed system as well, at least for authenticating to an Online Access Control System.
\subsubsection*{Coupling}
Section~\ref{sec:design-components} defines a number of individual components in the system, each component responsible for it's own, narrow task. A \acrshort{pdp} (only) evaluates policies, a User Directory (only) answers queries about users and the UserInfo endpoint only serves information about users to the external systems. This design represents a \textit{loosely coupled} system, where each module focuses only on its own task and is independent from its neighbours.

An alternative designed to the proposed system could be a single heavy-weight endpoint -- such as \acrshort{aaserver} -- that would take tasks of authenticating users, serving requests for user information and maybe even serving requests for protected resources. The advantages for external (and internal) applications would be that only a single endpoint is called at all times. Furthermore, only this single endpoint would be exposed to requests from the public internet, thus reducing the risks of security breach.

On the other hand, this approach suffers from the known disadvantages of tight coupling. The the single endpoint would be subject to bigger load and would need to be more versatile, than many separated endpoints. Furthermore, shall an outage occur to this single endpoint, the entire system would be inoperable, similarly, an update to a single functionality would require the update of the whole system. Loosely coupled, individual endpoints do not suffer from these pains. We maintain that benefits of the loose coupling outweigh its shortcomings and it presents a better fit in the access control scenario.
\subsubsection*{Future works}
As described in the beginning of this chapter, further work is needed to implement the envisioned the prototype and to fully implement the proposed system. Existing work on physical access control solutions with a smartphone should be reviewed in greater detail, before implementing this part of the system in practice. 

As new platforms announce support for FIDO2, the system flows should be redesigned to provide native support for this development, which could potentially simplify some of the authentication flows. Similarly, the implementation will be possible with less effort as the libraries mature and common practices are established around the specifications.

With further development of the system, additional use cases would likely have to be derived, to complement the minimal functionality of the proposed \acrshort{mvp}. During the development, several of such use cases were discovered, for example external employee, who needs access to the enterprise internal resources. In such case the system would need to be adapted to support identity federation across enterprises.