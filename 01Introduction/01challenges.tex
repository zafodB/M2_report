\subsection{Challenges} \label{challenges}

There is a number of challenges, which the current \acrshort{acs} in enterprises are facing. One of the major concerns of every access system is its security. Passwords are often targeted as the weakest link in the system. Every employee has their corporate account linked with many services and is accessing many protected resources. Often employees are forced to change their passwords periodically. Therefore, choosing a weak password and successful phishing attacks are a threat which can be solved by two-factor authentication, but as showed in \acrshort{nist} Special Publication 800-63B \cite{Grassi2017DigitalManagement} there are threats still associated with \acrshort{mfa} such as Social engineering, Phishing attacks or Endpoint compromise. Choosing a right combination of factors for authentication is therefore crucial.

Living in the era of fast technical advance, employees also require convenience when using access systems and carrying around an access card is not the most convenient method anymore, as shown in study by HID~\cite{2017AccessGlobal}, where 61\% of respondents sees integration between systems as hugely beneficial for user convenience or by Netflix~\cite{2012NetflixPilot}, where 87.5\% of respondents would like to use smartphones to open doors in the workplace for authentication. \acrshort{acs} should therefore, offer more convenient ways of authentication -- for example a smartphone.

Often, the physical access control and access policy management systems are two separate entities in enterprises. This requires more resources spend on managing these systems, as well as more set-up work when a new employee is hired or leaves the company. Having one system to handle these tasks is thus desirable.