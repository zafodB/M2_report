\subsection{Problem formulation} \label{problemFormulation}
The motivation for this project comes from the existing split of online and physical access control systems. Secondary, the low security of passwords has been demonstrated by several phishing cases and their use is discouraged.
% TODO add reference here
Alternative ways of strong authentication have been proposed instead, including physical cryptographic devices.
% TODO add reference here
The banking industry, which has been using similar tokens for a long period of time already (credit cards), has now witnessed the migration of these tokens to smartphones, mainly for user convenience.
% TODO add reference here
Such migration is also technically possible in access control scenarios, although its use has been limited in practice.

Taking into account the challenges and issues mentioned previously, we propose an innovative \acrshort{acs} that addresses these issues, by building on top of state-of-the-art technologies in the field. The problem formulation is as follows:

\begin{center}
    \textit{“How to design an Access control system, combining both physical and online access control, that includes strong authentication and enables the use of smartphone?”}
\end{center}

\paragraph{Sub-questions}
We define the following sub-problems to help guide us in the analysis and design of the system:

% TODO Is this still draft?
\begin{itemize}[noitemsep]
    \item What solutions are there on the market?
    \item What is the system architecture?
    \item Which technologies are the best for proposed system?
    \item Which components of the system must be implemented for MVP?
    \item What are the requirements of the system?
\end{itemize}