\subsubsection{OpenID Connect}\label{sec:oidc}

\acrfull{oidc} is an authentication layer standard developed by the OpenID Foundation. Current version is OpenID Connect 1.0, which has been proposed in November 2014~\cite{Sakimura2014Final:1}. \acrshort{oidc} is an extension and builds on top of the OAuth 2.0’s authorisation process, which is an authorisation standard and is described in Section \ref{sec:OAuth_2}. 

While OAuth 2.0 is designated for authorisation, it was frequently misused for authentication purposes by tweaking it and developing different extensions~\cite{RicherUser2.0}. Therefore, there was a need for simple authentication standard which would build on top of the OAuth, which led to introduction of \acrshort{oidc}.

The purpose of the \acrshort{oidc} is to allow the Client to authenticate the user, using the Authorisation server, which provides the identity information of the user. OAuth 2.0 is still used to obtain access tokens and use them to access resources which are protected~\cite{OpenIDSpecs}.

Consider example of a user using a user agent to access a web service which requires a log-in (this service is often called a \acrlong{rp} - RP). The user needs to log-in first, e.g. using Microsoft account. The user agent is redirected by the \acrshort{rp} to the \acrfull{op}, where the user has to enter her credentials and afterwards is requested to grant a permission for the service to access specific data from her Microsoft account. If the authentication of the user is successful, the \acrshort{op} provides the \acrfull{rp} with an \textit{id\_token} which verifies user’s identity and bears basic user’s information (e-mail, name, birthday, etc.). This \textit{id\_token} has the form of a \acrfull{jwt} which defines a way of conveying data as \acrshort{json} object between parties securely, compactly and self-contained. Other token which is obtained during the process is the \textit{Access Token}. User is then redirected back to the service. If the \acrshort{rp} needs to access protected resources, it uses the obtained \textit{Access Token} and the process continues using OAuth flow. This flow is explained in Section \ref{sec:OAuth_2}.