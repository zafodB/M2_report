\subsubsection{XACML}\label{sec:xacml}

 \acrfull{xacml} is a standardised, XML-based language for expressing security policies. It provides methods to define and combine policy sets and to rapidly identify, which policy applies to a given subject. \acrshort{xacml} was first standardised by OASIS\footnote{\url{https://www.oasis-open.org/}, accessed 04 March 2019} in 2003 and the latest version is XACML 3.0, from 2013~\cite{OASISStandard2013EXtensible3.0}. The further description is based on this latest version.
 
\acrshort{xacml} defines, how the \acrfull{pdp} which evaluates an input (access requests) against a policy set and issues an output -- a decision (such as \textit{access granted}). \acrshort{pdp} resides in an ecosystem comprised of other components that are necessary to maintain and enforce the access policy decisions (see Figure~\ref{fig:xacml-architecture} in Appendix~\ref{sec:analysis-access-policy} on page~\pageref{fig:xacml-architecture}). The other components in this ecosystem are out of scope of the \acrshort{xacml} specification~\cite{OASISStandard2013EXtensible3.0}.

\acrshort{xacml} defines the formal language of the policy set and of the requests/responses. A policy set contains one or more \textit{policies} and a \textit{policy combining algorithm}. The main components of a policy are:
% 
\begin{enumerate*}[label=(\roman*)]
    \item a \textit{target} to which this policy applies and
    \item one or more \textit{rules}.
\end{enumerate*} 
% 
The rule must specify a \textit{target} to which it applies and an \textit{evaluation} (\textit{permit} or \textit{deny}). It can also optionally specify \textit{conditions}, \textit{obligations} and \textit{advices}, all of which further shape the scope of the rule. Policy combining algorithm specifies the order and other conditions that decide, which policy will be finally applied on the request~\cite{OASISStandard2013EXtensible3.0}.
 
The \acrshort{xacml} standard defines in detail several logical entities and their interactions. The native format for messages between these entities is XML, but a profile has been developed to support JSON message format~\cite{2017JSON1.0}. Additional profile was also created to describe implementation in a RESTful architecture~\cite{2017REST1.0}.
 
 Several implementations of \acrshort{xacml} exist in Java, Python and other languages. Criticisms of \acrshort{xacml} include low adoption rate, unsuitability for federated enterprises~\cite{Cser2013XACMLDead}, and lack of transparency for the end user~\cite{Cser2013XACMLDead, Ardagna2011ExpressiveApplications}.

