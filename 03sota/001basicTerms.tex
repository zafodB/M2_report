\subsection{Basic terms}\label{sec:basic-terms}

\paragraph{Policy}
The IT portfolio of many current enterprises consists of a variety of different applications and systems. As a fine grained control of users' access to these resources is required, management of access rights for every user can become tedious. A set of rules is usually in place, which specify how a \textit{subject} (a person, user, actor) can interact with an \textit{object} (system, program), describing subject's accountability and capabilities within the object~\cite{Feltus2008PreliminaryConcept}. Policies are often used in the \hyperref[sec:xacml]{XACML} context.

\paragraph{Assertion}
An assertion is a ``confident and forceful statement of fact or belief''\footnotemark. In the context of authentication and in particular \hyperref[sec:saml]{SAML}, it is often the case that the identity provider is an independent entity, logically and physically separated from the application/relying party that requested user authentication. After the identity provider authenticates the user, it issues an assertion, confirming that the user's identity has been verified.

\footnotetext{\url{https://en.oxforddictionaries.com/definition/assertion}, accessed 05 March 2019}

\paragraph{Single Sign-on}
Single sign-on systems are such systems where the user only needs to authenticate once to be logged in to different services from different providers~\cite{Suoranta2014LogoutSolutions}. In practice, big common providers, such as Google or Facebook are used to sign the user in, and the user's profile with the provider is then used to access the content or services hosted by a third party.

\paragraph{Multi-factor authentication}
Multi-factor authentication is an approach where two or more factors for successful authentication of a user are required. This way, if one of factors is compromised, there is a need for compromising the other factors as well in order to successfully access the resources. \acrshort{nist} in their Digital Identity Guidelines~\cite{Grassi2017Digital3} specifies three groups of authenticators which can be used for authentication:
\begin{itemize}[noitemsep]
    \item Something you know (typically a password, \acrshort{pin}, etc.)
    \item Something you have (typically a physical cryptographic key, authenticator application, etc.)
    \item Something you are (typically a fingerprint, retina scan, etc.)
\end{itemize}

\paragraph{JWT}
\acrfull{jwt} \textit{``is an open standard (RFC 7519) that defines a compact and self-contained way for securely transmitting information between parties as a JSON object.''}\cite{JWT.ioJSONToken}. It is digitally signed using HMAC, RSA or ECDSA and therefore it can be trusted and verified providing the integrity of claims. Confidentially of the claim can be achieved by encrypting encryption of the token. \acrshort{jwt}s are often used for Information exchange or Authorisation (e.g. \acrshort{sso}). The structure of the \acrshort{jwt} is: header, payload and signature, all separated by a dot.~\cite{InternetEngineeringTaskForceIETF2015JSONRFC7519}