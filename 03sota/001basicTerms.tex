\subsection{Basic terms}

\paragraph{Policy}
The IT portfolio of many current enterprises consists of a variety of different applications and systems. As a fine grained control of users' access to these resources is required, management of access rights for every user can become tedious. A set of rules is usually in place, which specify how a \textit{subject} (a person, user, actor) can interact with an \textit{object} (system, program), describing subject's accountability and capabilities within the object~\cite{Feltus2008PreliminaryConcept}. Policies are often used in the \hyperref[sec:xacml]{XACML} context.

\paragraph{Assertion}
An assertion is a ``confident and forceful statement of fact or belief''\footnotemark. In the context of authentication and in particular \hyperref[sec:saml]{SAML}, it is often the case that the identity provider is an independent entity, logically and physically separated from the application/relying party that requested user authentication. After the identity provider authenticates the user, it issues an assertion, confirming that the user's identity has been verified.

\footnotetext{\url{https://en.oxforddictionaries.com/definition/assertion}, accessed 05 March 2019}