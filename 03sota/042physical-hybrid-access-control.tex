\subsubsection{Physical and hybrid access control -- HID Global} \label{sec:pacs-hid}

For the physical and hybrid access control, HID\footnote{\url{https://www.hidglobal.com/}, accessed 02 April 2019} is chosen with its HID Mobile access and Extended Access solutions. HID is one of the leading companies~\cite{TopManufacturers} on the market offering access controls, identity management or \acrshort{iot} solutions. Targeting mostly enterprises with the offer of the whole package, including software and hardware, tailored for the need of an enterprise. Both technologies we are about to explain are designed and implemented based on enterprise’s needs. Unfortunately, no detailed documents explaining the working of the system and technologies used in it has been found. Therefore, the following is a summary of what HID can offer with their system, which may be interesting for us and may be the source of inspiration. No in-depth technical look will be offered.

\paragraph{HID Mobile Access}
Mobile Access by HID\footnote{\url{https://www.hidglobal.com/solutions/hid-mobile-access-solutions}, accessed 02 April 2019} is a physical access solution for enterprises, which replaces, but can also supplement, the access cards. The solution focuses on the physical access mostly, where mobile devices such as smartphone, wearables or tablets can be used for authentication and granting the access. \acrshort{ble} and \acrshort{nfc} is used~\cite{HIDGlobal2014HIDGlobal} for communication with a reader which has to be mobile enabled. Devices with different operating systems (Android, iOS, BlackBerry) are supported as long as they are \acrshort{nfc} or \acrshort{ble} enabled. When accessing the premises, two modes of communication with a reader are offered: 
%
\begin{enumerate*}[label=(\roman*)]
    \item \textit{Twist\&Go} - where a device can be a few meters from a reader and has to be twisted to begin authentication utilising \acrshort{ble},
    \item \textit{Tap\&Go} - where a device has to be swiped over a reader to be authenticated using \acrshort{nfc}.
\end{enumerate*}
%
HID Secure Identity Service is used for managing employees’ Mobile \acrshort{id}s. Once a Mobile \acrshort{id} is created in the system, the employee receives an invitation with access codes, downloads the application, enrols into it and starts using it without further dues. As the case study at Netflix showed~\cite{2012NetflixPilot}, employees find mobile access intuitive and 90\% of surveyed employees found it easy to use.

\paragraph{Extended Access}
HID offers an Extended Access Technologies\footnote{\url{ https://www.hidglobal.com/extended-access-technologies}, accessed 02 April 2019} besides their physical access. The idea behind is to use the same credentials and ‘authenticator’ throughout the enterprise during the day instead of several, which is more convenient for the employee and offers wider range of applications for the enterprise. It also offers the policy management endpoint from which the accesses are granted. A Trusted Identity~\cite{2018ExperiencingPredictive} is used, which means that an employee’s ‘authenticator’ is verified once and can be used since freely. This Trusted Identity can be built into into smartcards, wearables or smartphone application allowing the use of printers, payment at the canteen, booking of meeting rooms, accessing premises, etc. All with a single device that has been once verified. In case of smartphone app and for a physical access to premises, every single employee can be individually recognised, even if a crowd is moving through a door. In case of higher security measures, multi-factor authentication can be enabled and biometric or \acrshort{pin} can be used. On top of that, the system provides advanced predictive analytics for better optimisation of a work place. Unfortunately, no additional information is shared by HID publicly, besides brochures, and therefore, we cannot look into what technologies they are using. The only obvious technologies are \acrshort{ble} and \acrshort{nfc}.