\subsection{Unified Modeling Language}
\acrfull{uml} is \textit{``A specification defining a graphical language for visualizing, specifying, constructing, and documenting the artifacts of distributed object systems.''}~\cite{OMG2017About2.5.1}. The latest version of the specification is \acrshort{uml} 2.5.1 which was published in 2017. It is describes the set of diagrams and their standardized notations which are being used in software system's and business modelling as well as other nonsoftware  systems. In total 14 \acrshort{uml} diagrams are defined. There are two principal types of diagrams: \textit{Behavior} and \textit{Structure} diagrams~\cite{ObjectManagementGroup2015Unified2.5}. Main difference between the two is that Behavior diagrams indicate system's objects dynamic behavior and Structure diagrams indicate their static structure. In the process, we have used one structure diagram - class diagram and two behavior diagrams - use case and sequence diagrams, which helped us to visualize flows within the system...

\subsubsection{Class diagram}
Class diagram is a static diagram which specify the system's structure. It is used for detailed modelling of the application's structure by visualising all classes, containing the name, attributes and operations of the class, and static relationships between them. Detailed and well designed class diagram can afterwords be translated to a programming code. Designing a class diagram ahead, before programming, helps us to structure a code better and saved us time during the actual programming phase.

\subsubsection{Use case diagram}
Use case diagram is a behavior diagram which describes the interaction of an external user with a system. Main components of a use case diagram are use cases, actors, associations. Actors are external users, humans or other systems, who are interacting with the system. Use cases depict a possible scenarios and use of a system which an actor can come across and are drawn in ovals. Associations are relations between use cases and actors. Use case diagram helps us to visualise all actors of our system, both humans and external systems, to know and see with which use cases are they interacting with, and where are the boundaries of our system.

\subsubsection{Sequence diagram}
Sequence diagram is a behavior diagram describing the interaction between objects and their message exchange in sequence. It consists of vertical lines known as lifelines which represent objects and processes that run in parallel, and horizontal lines which represents messages that are being interchanged in sequence. We use the sequence diagram to demonstrate message exchange between entities of our system in a few basic scenarios.