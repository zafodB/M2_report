\subsection{Minimum Viable Product}
The \acrfull{mvp} is described as the minimal set of features that a customer would be willing to pay for~\cite{SteveBlank2010PerfectionFeatureSet}. It does not describe a desired final state of the product, rather it describes a first marketable version. The first version is only aimed at a fraction of the market, at customers known as \textit{early adopters}, who are more forgiving towards lower utility and are more likely to provide early feedback on the product. If any one feature would be removed from the \acrshort{mvp}, the product would have such a limited utility, that it would be very unlikely adopted by any customers, even the early adopters~\cite{Junk2000TheProjects}.

While this project does not examine the market for this solution, we use the \acrshort{mvp} methodology to focus on most important product features in the requirement elicitation and design stages. The \acrshort{mvp} approach helps us identify, which features need to be present in the prototype and the first version of the product, and to differentiate the product developed in this project from the full-featured product, which would be achieved if the project continued further.