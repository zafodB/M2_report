\subsection{Software Prototyping}\label{sec:methodology-prototype}

A prototype is a model, an initial version of the intended system. The goal of the prototype is to demonstrate the final product, present the idea, test out design choices or compare two variants of the product. It is cheaper and faster to produce a prototype than it is to produce a fully functional system, so the prototype can also be used to reduce risk from uncertain requirements~\cite{Sommerville2011SoftwareEngineering, Davis1995SoftwarePrototyping}.

It is argued, that the purpose of the prototype needs to be determined early for the prototype to bring value and that the prototype must be evaluated after it is implemented. In this project, the prototype is implemented to demonstrate the feasibility of the technology in the access control scenario. It achieved by implementing a subset of the requirements in an experiment prototype. Experiment prototype verifies a possible solution to user needs, but leaves out the user interface choices and non-functional requirements (such as speed or reliability). This is contrasted by exploratory prototype (used to discover new requirements) and evolution prototype (prototype that is iteratively evolved into the final product)~\cite{Davis1995SoftwarePrototyping}.