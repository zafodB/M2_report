\subsection{OpenID Connect}

\acrfull{oidc} is an authentication layer standard developed by OpenID Foundation. Current version is OpenID Connect 1.0, which has been proposed in November 2014.(REF) \acrshort{oidc} is an extension and builds on top of the OAuth 2.0’s authorization process, which is an authorization standard and has been described in Section \ref{OAuth_2}. 

As already mentioned, OAuth 2.0 is designated for authorization but at the time it has been ‘misused’ for authentication purposes by tweaking it and developing different extensions (REF! User Authentication with OAuth 2.0) . Therefore, there was a need for simple authentication standard which would build on top of the OAuth and that is \acrshort{oidc}.

The purpose of the \acrshort{oidc} is to allow the Client to authenticate (verify the identity) of the user, using the Authorization server, which also provides identity information of the user, while OAuth 2.0 is being used to obtain access tokens and use them to access resources which are protected.(REF) Technically speaking \textit{“OpenID Connect specifies a RESTful HTTP API, using JSON as a data format”}.  (XYZ-quote)

NOT COMPLETELY TRUE, REWRITE

Lets take an example of an user accessing a web service which requires log-in (also called \acrfull{rp}). The user needs to log-in first, e.g. using \acrfull{ms} account. User is redirected to the \acrfull{op}, where he has to enter his credentials and afterwards is requested to grand a permission for the service to access specific data from the \acrshort{ms} account, this happens on the Authorization server. If the authentication of the user is successful, \acrshort{op} contacts \acrfull{idp} and obtains \textit{id\_token} which verifies user’s identity and bears basic user’s information. This \textit{id\_token} is in form of \acrfull{jwt} which defines a way of conveying data as \acrshort{json} object between parties securely, compactly and self-contained. Other tokens which are obtained during the process are \textit{authorization\_code} and Access Token. User is then redirected to post login \acrshort{url}. Process then continues, but it is the OAuth which does the work further.