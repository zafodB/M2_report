\subsection{OAuth 2.0} \label{OAuth_2}

Before existence of OAuth, the access to user's protected resources was commonly delegated to a third party by sharing the user's credentials with that third party. This approach has flaws, since it is not possible for the user to fine-tune the access to their account, neither across different third parties, nor for different access levels of any given third party. 

The OAuth 1.0 was published in 2010 as to address this problem 
% REFERENCE The OAuth 1.0 Protocol
and was followed in 2012 by OAuth 2.0. 
% REFERENCE The OAuth 2.0 Authorization Framework
OAuth 2.0 is a Proposed Standard, is not compatible with version 1.0 and is widely used today. In the following sections of this report we will refer to OAuth 2.0.

In OAuth, the third party requesting access to user's protected resources is known as the \textit{client}. The server controlling the protected resources has an authorisation server associated with it. To gain access to the resources, the client needs to first liaise with the authorisation server to obtain an \textit{access token} (also known as the bearer token). This token is then used to access the resource.

% TODO visio diagram of Authorisation code flow https://speakerdeck.com/nbarbettini/oauth-and-openid-connect-in-plain-english?slide=10

% TODO describe the diagram and the process here

\paragraph{Scopes}
todo
% TODO something about scopes here

The process outlined above is mostly concerned with authorisation of the client to access protected resources. However, it emerged as a common practice
% REFERENCE https://oauth.net/articles/authentication/
that OAuth was used to authenticate users. This is done so that the client is only authorised to read a scope containing basic user data. While this is technically possible, further specifications were developed to standardise this usecase. We cover these in the following section.
% FINAL check if following section is about OpenID Connect