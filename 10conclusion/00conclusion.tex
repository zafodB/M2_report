\section{Conclusion}\label{sec:conclusion}

Access control solutions nowadays are often separated. One system handles physical security, such as doors and checkpoints, and another independent system handles signing in online. Furthermore, the systems usually offer limited support for mobile devices. The aim of this project is to design and implement a system that combines physical and online access control.

First, the methods and processes used in this projects are described in the Methodology chapter. State of the Art considers two existing solutions are examined in the State of the Art, followed by significant technologies for authentication, authorisation and policy management. 

In the Analysis chapter, the access control systems are further classified into categories and the different goals of each category are outlined. Considerations for the combined system are made from different angles and most suitable technologies are suggested on a high level. Use cases are drawn from previous considerations to best address the problem specified by the problem formulation. Based on the use cases, functional and non-functional requirements are derived. Both the use cases and the requirements are developed for the system as an \acrshort{mvp}, Use cases that are not critical for the system to perform its functions were deliberately not included to maintain focus on the vital characteristics.

System architecture and entities are described in the System Design chapter. They are followed by detailed explanation of four use cases, that were selected to be implemented in the prototype. Flows in the system during sign-in, accessing protected resources and policy evaluation conclude the System Design chapter.

The prototype implementation consists of two parts, one focusing on the FIDO flow and the other one on OAuth. Setup and used libraries and frameworks are described, followed by short evaluation. Not all selected use cases were successfully implemented, but the prototype is still deemed valuable, as it does present two of the most crucial flows of the system. 

This report gradually answer the questions set in the Problem Formulation. The State of the art chapter laid a basis for knowledge about current technologies which were analysed in the Analysis chapter, where the evaluation has been done and the most relevant ones were suggested for the implementation. Combination of \acrlong{oidc} and OAuth 2.0 has chosen for the Authorisation, FIDO2 for Authentication and \acrshort{xacml} for Access Policy Management purposes.

Knowing possible scenarios of actors' interaction with a system is crucial for the precise definition of requirements. All requirements were defined for the product as \acrlong{mvp} with the most focus put towards functional ones. These were then narrowed down to define a prototype's functionality for implementation. Based on them, the system's architecture and access control flows were defined and system components identified. Even-thought, this was done for the prototype definition, the \acrshort{mvp} would have very similar architecture, components and access control flows.

Regarding the support of both, smartphones and physical tokens, we demonstrated the registration process carried out on laptop or smartphone with the use of physical cryptographic key and subsequently signing-in on the other device, as well as registration process on the smartphone using the built-in authenticator. The one limitation imposed as of now, is the possibility of registration of only one authenticator, either the cryptographic key or built-in authenticator, not both.