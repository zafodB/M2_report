\subsection{Access Policy Evaluation}\label{sec:analysis-access-policy}

In every \acrlong{acs}, the model used for granting the access play a crucial role as the system is dependent on it. Often it makes it easier for the administrator to manage and grant the access to employees, but if a wrong model is chosen or is wrongly implemented, it can make the work even more burdensome. As it was outlined in Section~\ref{sec:sota} (page~\pageref{sec:sota}), we are looking into \acrshort{rbac} and \acrshort{abac} as models for Identity and Access Management as they are currently the most used ones~\cite{2018BestV3}.

Both models offer a range of functionality and suits specific enterprises. One of \acrshort{rbac}’s big advantages is the ease of creating and maintaining the roles and system as a whole, but as the enterprise grows and more roles and resources are introduced, it gets complicated and hard to have an overview. Therefore, \acrshort{rbac} is suitable for small to medium enterprises. It is less complex than \acrshort{abac} and therefore it offers low level of granularity. Permission only can be granted to roles, not to operations or objects. On-the-fly contextual decisions are not supported, as well as restrictions can only be applied to parts of the system, not on specific data. Also, only known parameters can be used during implementation and environmental restrictions cannot be implemented. Even though, \acrshort{rbac} is older model it still offers adequate functionality.

On the other hand, \acrshort{abac} model which is based on attributes and policies presents fine grain and multi-dimensional access control. Biggest advantages of \acrshort{abac} over \acrshort{rbac} are the scalability of the model, dynamic parameters, easy maintenance and support for on-the-fly context aware decisions as information about requesting subject, requested object and environmental attributes are present. This way it is possible to grant access to specific data at specific times based on location. Initial configuration of the system is more complex compared to \acrshort{rbac}, but once done, it is easy to add new attributes or policies which are automatically executed. 

As the focus of the project, is to offer \acrlong{acs} to big enterprises, allowing fine grain control over resources, \acrlong{abac} model is more suitable for the solution. The trend in the industry is to adopt \acrshort{abac} more and more, and Gartner predicts that \textit{``By 2020, 70\% of businesses will use attribute-based access control (\acrshort{abac}) to protect critical assets''}~\cite{GartnerGartnerPredictions}. Studies also shows, that \acrshort{abac} is the model which should be used by enterprises in the future, rather than \acrshort{rbac}~\cite{Fatima2016TowardsArgument}.

To implement \acrshort{abac} model, the standard have to chosen as well. The most wide-spread standard is \acrshort{xacml}, which is explained in Section \ref{sec:xacml}, but others such as \acrfull{alfa} or \acrfull{ngac} exists as well. \acrshort{alfa} is a pseudocode language based on \acrshort{xacml} which provide higher user-friendliness, lower complexity and overall similar performance~\cite{Mejri2016FormalPolicies}. \acrshort{ngac} on the other hand, even-though it uses \acrshort{abac} model for access control, its implementation differs from \acrshort{xacml}~\cite{Ferraiolo2016ANGAC}. For the purpose of this project, we decide and propose the use of \acrshort{xacml} as the standard for implementing \acrshort{abac} in the system, mainly because of its wide spread, similar functionality to \acrshort{ngac} and the fact that it is being taught at the Identity and Access Management course at the university.

% TODO write about OAuth client credentials use case - how the application can access its own resources
% TODO distinguish between user and application access