\subsection{Access Policy Evaluation}\label{sec:analysis-access-policy}

In every \acrlong{acs}, the model used for granting the access plays a crucial role as the system is dependent on it. If a wrong model is chosen or is wrongly implemented, it can make the task of access granting burdensome. As outlined in Sections~\ref{sec:sota-rbac} and \ref{sec:sota-abac}, \acrshort{rbac} and \acrshort{abac} are currently the most used models for Identity and Access Management~\cite{2018BestV3}.

Both models offer a range of functionality and cater to specific enterprise needs. One of the big advantages of \acrshort{rbac} is the ease of creating and maintaining the roles and system as a whole. However, but as the enterprise grows and more roles and resources are introduced, complexity of the system grows as well, and it can become difficult to maintain an overview. Therefore, \acrshort{rbac} is more suitable for small to medium enterprises. It is less complex than \acrshort{abac} and it offers lower level of access granularity. Permissions only can be granted to roles, not to operations or objects. On-the-fly contextual decisions are not supported and restrictions can only be applied to parts of a system, not on specific data. Furthermore, only known parameters can be used during implementation and environmental restrictions cannot be implemented. Even though, \acrshort{rbac} is an older model it still offers adequate functionality in some scenarios.

On the other hand, \acrshort{abac} model which is based on attributes and policies presents fine-grained and multi-dimensional access control. Biggest advantages of \acrshort{abac} over \acrshort{rbac} are the scalability of the model, dynamic parameters, easy maintenance and support for on-the-fly context aware decisions, as information about requesting subject, requested object and environmental attributes are present. This way it is possible to grant access to specific data at specific times based on location or other attributes. Initial configuration of the system is more complex compared to \acrshort{rbac}, but once done, it is easy to add new attributes or policies which are automatically executed. 

As the focus of the project, is to offer \acrlong{acs} to big enterprises, allowing fine-grained control over resources, \acrlong{abac} model is more suitable for the solution. The trend in the industry is to adopt \acrshort{abac} more and more, and Gartner predicts that \textit{``By 2020, 70\% of businesses will use \acrshort{abac} to protect critical assets''}~\cite{GartnerGartnerPredictions}. Studies also show, that \acrshort{abac} is the model which should be used by enterprises in the future, rather than \acrshort{rbac}~\cite{Fatima2016TowardsArgument}.

To implement the \acrshort{abac} model, the particular technologies need to chosen as well. The most wide-spread standard is \acrshort{xacml} (Section \ref{sec:xacml}), but others such as \acrfull{alfa} or \acrfull{ngac} exists as well. \acrshort{alfa} is a pseudocode language based on \acrshort{xacml} which provide higher user-friendliness, lower complexity and overall similar performance~\cite{Mejri2016FormalPolicies}. \acrshort{ngac} on the other hand, even-though it uses \acrshort{abac} model for access control, its implementation differs from \acrshort{xacml}~\cite{Ferraiolo2016ANGAC}. For the purpose of this project, we propose the use of \acrshort{xacml} as the technology for implementing \acrshort{abac} in the system, mainly because of its wide spread and similar functionality to \acrshort{ngac},

% TODO write about OAuth client credentials use case - how the application can access its own resources
% TODO distinguish between user and application access