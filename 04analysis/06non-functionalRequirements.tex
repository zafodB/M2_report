\subsubsection{Non-functional requirements} \label{sec:analysis_Non-functional}
Non-functional requirements describe what should a system be and how should it perform. They can be evaluated based on measures. User usually cannot directly see their implementation but can rather feel it by the performance. Often the architecture of the system is influenced by them. The list of non-functional requirements along with requirement’s ID, description and reference to where it was gotten from can be found in Table \ref{tab:nonfunctional_requirements}.

\begin{table}[H]
    \onehalfspacing
    \centering
    \begin{tabular}{|c|p{12cm}|c|}
    \hline
    \cellcolor[HTML]{CBCEFB}\textbf{ID}&\cellcolor[HTML]{CBCEFB}\textbf{Description}&\cellcolor[HTML]{CBCEFB}\textbf{Reference}\\
    \hline
    RQ-20&Authentication to PACS using enterprise digital identity must be comparably fast to traditional RFID access cards (order of milliseconds to seconds).&UC-2\\
    \hline
    RQ-21&External system must be able to access protected resources.&-\\
    \hline
    RQ-22&The system must use OIDC + OAuth for Authentication and Authorisation.&\makecell{Section \ref{sec:analysis-authentication}\\and \ref{sec:analysis-authorisation}}\\
    \hline
    RQ-23&The system must use XACML for Identity and Access management.&Section \ref{sec:analysis-access-policy}\\
    \hline
    \end{tabular}
    \caption{List of non-functional requirements for the \acrshort{mvp} with the reference of where they are collected from.}
    \label{tab:nonfunctional_requirements}
\end{table}