\subsubsection{Non-functional requirements} \label{sec:analysis-nonfunctional}
Non-functional requirements describe how the system is and how does it perform. They can be evaluated based on measures~\cite{Adams2015NonfunctionalDesign}. User usually does not directly see their implementation, but can rather feel it by the system performance. Often they influence the architecture of the system. Table~\ref{tab:nonfunctional-requirements} lists these requirements for the proposed system. The \textit{Reference} column indicates the origin of the given requirement to map, why that requirement has been included.

\bigskip\noindent
In this chapter we first define four categories of access control problems. We continue by considering the access control problem and possible solutions from the perspectives of authentication, authorisation and access policies. We then identify the basic use cases of a combined access control system and use these to derive the functional and non-functional requirements. In the next chapter we describe the design of the system in greater detail.