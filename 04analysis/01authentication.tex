\subsection{Authentication}\label{sec:analysis-authentication}
In \acrshort{pacs}, physical tokens, such as ID cards, are the common practice nowadays. A survey carried out in 2016 indicated that above 40\% of the companies use low frequency proximity cards as their main method for personal authentication~\cite{HIDGlobal2017TheEnterprise}. These cards, operating in 125-134kHz frequency range are considered an obsolete technology and are not secure~\cite{Hakamaki2015SecurityTechnology}. 

On the other hand, around 20\% of the companies indicated use of mobile devices as a part of their \acrshort{pacs} solution~\cite{HIDGlobal2017TheEnterprise}. The main benefit of using a mobile device for this purpose is, that most people already own a smartphone and usually carry it with them at all times. Other technologies represented in the survey that are considered secure today, include iCLASS (contact-less card) and MIFARE DesFire (contact-less card). FIDO keys (NFC, Bluetooth, or other) are not mentioned in the survey, although the UAF and U2F specifications were published (as proposed standards) in 2014 and FIDO2 was first published (also only as a proposed standard) in 2015. Today, compatible \acrshort{pacs} systems exist on the market\footnotemark. 

FIDO and FIDO2 enjoy support of big companies like Google, Facebook and Microsoft, but consumer-oriented services from these providers only permit the use of FIDO/FIDO2 as the second authentication factor. Google uses FIDO U2F keys for their employees, but only in the online login scenario as a password replacement~\cite{Krebs2018Google:Phishing} and Microsoft only supports FIDO2 keys with personal Microsoft accounts. As shown in Section~\ref{sec:online-access-control}, even Azure AD does not support this technology. To explore how FIDO2 can be used for hybrid access control on an enterprise scale, we propose the support of FIDO2 as a requirement for our system.
% 
\footnotetext{\url{https://web.archive.org/web/20190319221618/https://www.yubico.com/works-with-yubikey/catalog/modis/}, accessed 19 March 2019}

There is already a variety of devices that have been FIDO2 certified, with the Android system being a recent addition to the list~\cite{FIDOAlliance2019AndroidPasswords}. Considerations need to be put into how the FIDO2 flow is incorporated in the system and which of the FIDO2-certified devices are used in the flow. In \acrshort{pacs}, usability and speed are important criteria, since existing systems already offer very low- to no-latency during a typical door opening use case. On top of physical access control, the current access management solutions are mostly used for employee identification (as an ID badge)~\cite{HIDGlobal2017TheEnterprise} and the employee is only issued a single badge. The recommended approach in case the employee forgets the badge is to assign a temporary one that can be used during a limited time period and must be returned afterwards~\cite{Ryan2018HowBadges}.

In the online authentication, we can see that system supports a wider range of authentication (and account recovery) methods, and the user is left to decide, which is authentication method do they prefer (see Section~\ref{sec:online-access-control}). Offering multiple methods simultaneously makes sense in the online access control scenario, because unlike in \acrshort{pacs}, where the on-site reception/security personnel can verify employees' identity and issue a temporary badge, in the online access control the identity verification is more complicated. In the case of forgotten credentials, the identity would therefore needed to be verified remotely (typically over phone), which is not feasible.

In the proposed hybrid access control system, it is desired to provide an authenticator, that is both fast to use and can be easily complemented and/or substituted with another authentication method if lost or forgotten. We propose the combination of FIDO2 key with \acrshort{nfc} as the primary authenticator, and FIDO2 enabled smartphone as the supplementary authenticator to meet the above criteria. The FIDO2 key with NFC offers proximity-card-like usability (the user needs to swipe/hold the key next to the reader). If lost or forgotten, it can be substituted by an FIDO2 enabled smartphone, as people generally carry a smartphone with them, while at work. During our research, we have not encountered systems which offer the same selection of technologies, although the combination of a smartphone and an access card is presented in Section~\ref{sec:pacs-hid}.

% TODO FIDO2 would be used for all 4 boxes from access control breakdown chart

% TODO How OpenID connect is used for authentication/SSO
% TODO What information needs to be shared (Default OIDC scope - name, email, UID)...?
% TODO What entities will be needed (OID provider)
% TODO OpenID connect would be used only for boxes 3 and 4 from access control breakdown chart


% For \acrlong{aal}s 2 and 3 defined in 
% %  REFERENCE SECURITY REQUIREMENTS FOR CRYPTOGRAPHIC MODULES
% , authentication must be carried out using at least two independent factors. This requirement, together with abandonment of memorised secrets results in adoption of new authenticators. These  authenticators should offer at better security as memorised secrets, while maintaining the same level of accessibility. In 
